% Usar el tipo de documento: Artículo científico.
\documentclass[11pt,a4paper]{article}

% Cargar mensajes en español.
\usepackage[spanish]{babel}

% Usar codificación utf-8 para acentos y otros.
\usepackage[utf8]{inputenc}
\usepackage[T1]{fontenc}
\usepackage{lmodern}

%Dimensiones de los márgenes.
%\usepackage[margin=1.5cm]{geometry}

% Insertar porciones de código
\usepackage{listings}

% Comenzar párrafos con separación no indentación.
\usepackage{parskip}
%enlaces
\usepackage{hyperref}
% Usar gráficos
\usepackage{graphicx}
\usepackage{caption}
\usepackage{subcaption}
%
% Usar contenedores flotantes para figuras.
\usepackage{float}

% Carpeta de las imágenes.
\graphicspath{{img/}}



%Gummi|065|=)
\title{\textbf{Bienvenido al futuro}}
\author{rodrigo.arias@udc.es}
\date{}




\begin{document}

\maketitle

\section{Expectantes}

Al fin y al cabo un videojuego es la realización más atractiva que uno pudiera 
plantearse al producir gráficos en un ordenador.

Introduce al sujeto en una nueva realidad, provocando que experimente nuevas 
situaciones que han sido cuidadosamente elaboradas. Con el mero propósito de 
causar una sensación sin igual, sin precedentes.

Las cualidades son bien conocidas. Lugares exóticos que no podrían ocurrir en el 
mundo actual. Al menos no en nuestra época. Un sueño donde somos conscientes.  

A todos nos encanta viajar. Experimentar las costumbres de otros lugares.  
Olvidarnos por un segundo de nuestra pequeña burbujita y emprender una aventura.  
Sin conocer el final. Expectantes.

Entonces la modernidad de una época, junto a la intriga más profunda, trastoca 
el significado de la antigua palabra para incluirla en el eje temporal. Que 
ocurriría si pudiesemos hacerlo. Viajar en el tiempo.

Paradojas, dicen los expertos. Pero eso no sacia nuestra curiosidad. No nos 
detiene. Veámoslo con nuestros propios ojos.

\section{La máquina}

Es preciso una máquina. Un dispositivo no mucho más grande que una caja donde 
cabe una persona. Un temporizador permite ponerla en funcionamiento de forma programada.

El funcionamiento es bastante sencillo. Aunque sus consecuencias no lo son 
tanto.

Primero activa el temporizador para programar el encendido de la máquina.  
Debes anotar el momento exacto en el que la máquina será activada. Llamaremos a 
este instante el momento A. Asegúrate de que tienes tiempo suficiente para salir 
de ahí, pues no quisieras encontrarte con quien tú sabes.



Aléjate, y realiza las tareas que desees mientras dejas que la máquina se encienda sola.  
Cuando estés listo, prepárate para entrar en la misma. Anota el momento exacto 
en el que entras en la máquina. Este será el momento B.

La máquina te llevará de vuelta al momento A. En ese instante existirán dos 
versiones de tí mismo.

Mantén especial cuidado en no modificar la conducta que has realizado antes de 
entrar en la máquina. Si lo haces, las consecuencias son inciertas. 

\section{Consecuencias}

Y entonces, ¿que ocurriría? El juego permite realizar viajes en el tiempo y observar los efectos que provoca. Permite aprovechar las ventajas de viajar en el tiempo, para poder resolver una serie de escenas.

En cada escena deberás alcanzar un lugar determinado para continuar. Para ello, 
utiliza los objetos que encuentres, y usa tu imaginación e ingenio para resolver 
el rompecabezas.

\begin{figure}[htp]
\centering
\includegraphics[scale=0.7]{/home/rodrigo/fic/hot/ciie/2d/boceto1.png}
\caption{Boceto de un nivel}
\label{}
\end{figure}

En la figura se puede ver el ejemplo del diseño de un nivel.

1. El personaje se encuentra en una sala, con una caja, y una salida. Aunque se suba a la caja, no consigue llegar a la salida, le hace falta otra caja.

2. Coloca la caja debajo de la salida.

3. Acciona el temporizador de la máquina.

4. Se aparta de la máquina y se mantiene en la última habitación hasta 9.

5. La máquina se enciende de forma programada.

9. Coge la caja, y se mete con ella en la máquina. Ésta lo teletransporta a 5, el momento en el que se encendió la máquina.

6. Utiliza la caja que acaba de traer, para acceder a la salida.

7. Se sube a las dos cajas y sale.

8. Se lleva la caja que ha traído, pues debe dejar la escena de la misma forma que estaba.

10. Coge la caja, y se mete en la máquina.


\end{document}
